% -----------------------------------------------------------------------------
% Modelo de apresentação de trabalho acadêmico
% Projeto hospedado em: https://github.com/cfgnunes/latex-slides
%
% Autor: Cristiano Fraga G. Nunes <cfgnunes@gmail.com>
% -----------------------------------------------------------------------------

\documentclass[aspectratio=43, 12pt]{latex-slides}
%\documentclass[aspectratio=43, 12pt, faixa_superior]{latex-slides}

\usepackage[%
    alf,
    abnt-emphasize=bf,
    bibjustif,
    recuo=0cm,
    abnt-doi=expand,       % Expande endereço DOI para http://dx.doi.org/
    abnt-url-package=url,  % Utiliza o pacote url
    abnt-refinfo=yes,      % Utiliza o estilo bibliográfico abnt-refinfo
    abnt-etal-cite=3,
    abnt-etal-list=3,
    abnt-thesis-year=final
]{abntex2cite}             % Configura as citações bibliográficas

% -----------------------------------------------------------------------------
% Pacotes utilizados
% -----------------------------------------------------------------------------
\usepackage[utf8]{inputenc}         % Codificação do documento
\usepackage[T1]{fontenc}            % Seleção de código de fonte
\usepackage{booktabs}               % Réguas horizontais em tabelas
\usepackage{color, colortbl}        % Uso de cores
\usepackage{graphicx}               % Inclusão de gráficos
\usepackage{indentfirst}            % Recua o primeiro parágrafo de cada seção
\usepackage{microtype}              % Melhora a justificação do documento
\usepackage{makecell}               % Tabelas com múltiplas linhas e colunas
\usepackage{multirow, multicol}     % Layout em múltiplas linhas e colunas
\usepackage{verbatim}               % Exibe texto tal como escrito no documento
\usepackage{icomma}                 % Vírgulas em expressões matemáticas
\usepackage{subeqnarray}            % Subenumeração de equações
\usepackage{amsmath}                % Funções matemáticas
%\usepackage[charter]{mathdesign}   % Utiliza a fonte 'Charter BT'
%\usepackage{newtxtext, newtxmath}  % Utiliza a fonte 'Times New Roman' (clone)
%\usepackage{palatino}              % Utiliza a fonte 'Palatino' (clone)
%\usepackage{lmodern}               % Utiliza a fonte 'Computer Modern'
%\usepackage{amsfonts}              % Fontes e símbolos matemáticos
%\usepackage{latexsym}              % Mais símbolos matemáticos
%\usepackage{xfrac}                 % Escreve frações de forma compacta

\usepackage{caption, subcaption}    % Caption de objetos
\usepackage{leading}                % Espaçamento entre linhas

% Utiliza a fonte 'Arial' (clone)
\usepackage[scaled]{helvet}
\renewcommand*\familydefault{\sfdefault}

% -----------------------------------------------------------------------------
% Preâmbulo
% -----------------------------------------------------------------------------

\title{Título do trabalho}
\author{Cristiano Fraga Guimarães Nunes}
\institute[CEFET-MG]{%
    \par\vspace{1em}
    Centro Federal de Educação Tecnológica de Minas Gerais
    \par\vspace{1em}
}
\date{01 de janeiro de 2020}

% -----------------------------------------------------------------------------
% Configurações
% -----------------------------------------------------------------------------

% Define as cores utilizadas no documento
\definecolor{blue_link}{RGB}{0,80,128}

% Configuração dos links do PDF (pacote: 'hyperref')
\makeatletter
\hypersetup{%
    portuguese,
    colorlinks=true,        % true: links coloridos; false: links em caixas
    linkcolor=black,        % Cor dos links internos
    citecolor=black,        % Cor dos links para as citações
    filecolor=black,        % Cor dos links para arquivos
    urlcolor=black,         % Cor dos links de URLs
    breaklinks=true,
    pdftitle={\@title},
    pdfauthor={\@author},
    pdfsubject={},
    pdfkeywords={}
}
\makeatother

% Configura o label das figuras
\captionsetup[figure]{labelformat=simple}

% Para justificação global
\renewcommand{\raggedright}{\leftskip=0pt \rightskip=0pt plus 0cm}

% -----------------------------------------------------------------------------
% Início do documento
% -----------------------------------------------------------------------------
\begin{document}

% Configura o espaçamento entre linhas
\leading{1.5em}

% -----------------------------------------------------------------------------
% Capa
% -----------------------------------------------------------------------------

\begin{frame}
    \titlepage
\end{frame}

% -----------------------------------------------------------------------------
% Sumário
% -----------------------------------------------------------------------------

\begin{frame}
    \frametitle{Sumário}

    \tableofcontents
\end{frame}

% -----------------------------------------------------------------------------
% Introdução
% -----------------------------------------------------------------------------

\section{Introdução}

\begin{frame}
    \frametitle{Introdução}

    \begin{itemize}
        \item Inserir seu texto aqui...
        \item Inserir seu texto aqui...
    \end{itemize}
\end{frame}

% -----------------------------------------------------------------------------
% Caracterização do problema
% -----------------------------------------------------------------------------

\begin{frame}
    \frametitle{Caracterização do problema}

    \setcounter{figure}{3}
    \begin{figure}[!t]
        \centering
        \includegraphics[width=0.55\textwidth]{./figuras/figura-exemplo1}
        \caption{Exemplo de figura.}
        \label{fig:exemplo_figura}
    \end{figure}
\end{frame}

\begin{frame}
    \frametitle{Caracterização do problema}

    \begin{itemize}
        \item Inserir seu texto aqui... \cite{nunes2017local}
        \item Inserir seu texto aqui...
    \end{itemize}
\end{frame}

% -----------------------------------------------------------------------------
% Motivação
% -----------------------------------------------------------------------------

\begin{frame}
    \frametitle{Motivação}

    \begin{itemize}
        \item Inserir seu texto aqui...
        \item Inserir seu texto aqui...
    \end{itemize}

\end{frame}

% -----------------------------------------------------------------------------
% Trabalhos Relacionados
% -----------------------------------------------------------------------------

\section{Trabalhos Relacionados}

\begin{frame}
    \frametitle{Trabalhos Relacionados}

    \begin{itemize}
        \item Inserir seu texto aqui...
        \item Inserir seu texto aqui...
    \end{itemize}
\end{frame}

% -----------------------------------------------------------------------------
% Fundamentação Teórica
% -----------------------------------------------------------------------------

\section{Fundamentação Teórica}

\begin{frame}
    \frametitle{Fundamentação Teórica}

    \begin{itemize}
        \item Inserir seu texto aqui...
        \item Inserir seu texto aqui...
    \end{itemize}
\end{frame}

\begin{frame}
    \frametitle{Métricas de avaliação}

    \begin{itemize}
        \item Inserir seu texto aqui...
        \item Inserir seu texto aqui...
    \end{itemize}
\end{frame}

% -----------------------------------------------------------------------------
% Metodologia
% -----------------------------------------------------------------------------

\section{Metodologia}

\begin{frame}
    \frametitle{Metodologia}

    \begin{itemize}
        \item Inserir seu texto aqui...
        \item Inserir seu texto aqui...
    \end{itemize}
\end{frame}

% -----------------------------------------------------------------------------
% Resultados
% -----------------------------------------------------------------------------

\section{Resultados}

\begin{frame}
    \frametitle{Base de dados para avaliação}

    \begin{itemize}
        \item Inserir seu texto aqui...
        \item Inserir seu texto aqui...
    \end{itemize}
\end{frame}

% -----------------------------------------------------------------------------
% Conclusão
% -----------------------------------------------------------------------------

\section{Conclusão}

\begin{frame}
    \frametitle{Conclusão}

    \setcounter{figure}{3}
    \begin{figure}[!t]
        \centering
        \includegraphics[width=0.55\textwidth]{./figuras/figura-exemplo2}
        \caption{Outro exemplo de figura.}
        \label{fig:outro_exemplo_figura}
    \end{figure}
\end{frame}

% -----------------------------------------------------------------------------
% Referências
% -----------------------------------------------------------------------------

\begin{frame}[allowframebreaks]
%\begin{frame}<presentation:0>[noframenumbering]
    \frametitle{Referências}

    \scriptsize
    \bibliography{referencias.bib}
\end{frame}

% -----------------------------------------------------------------------------
% Capa
% -----------------------------------------------------------------------------

\begin{frame}
    \titlepage
\end{frame}

\end{document}
